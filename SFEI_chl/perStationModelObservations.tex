
\documentclass[12pt]{amsart}
\usepackage{geometry} % see geometry.pdf on how to lay out the page. There's lots.
\geometry{a4paper} % or letter or a5paper or ... etc
% \geometry{landscape} % rotated page geometry

% See the ``Article customise'' template for come common customisations

\title{Observations from Per Station Models}
\author{Sara Stoudt}
%\date{} % delete this line to display the current date

%%% BEGIN DOCUMENT
\begin{document}

\maketitle
%\tableofcontents

\section{C10}
\subsection{Data}
\begin{itemize}
\item southmost station
\item peaks before 1975, 1976, 1991,1992, 2008, 2009
\item can clearly see year structure except for 1982, 1983,1995, 1998,2005, 2006, 2012
\end{itemize}

\subsection{Parsimonious Model}

\begin{itemize}
\item day of year captures clear year structure
\item pheo also helpful with clear year structure, magnitude reflects magnitude of chl
\item tn not super helpful
\item do.per like pheo although a little less consistent with the magnitude matching
\end{itemize}

\subsection{Full Model}

\begin{itemize}
\item sio2 no real variation 
\item same for tp and tss
\item nh4: a bit of peakedness, seems to lag a bit from the peak of chl, relationship lost after 1995

\end{itemize}

\section{C3}
\subsection{Data}
\begin{itemize}
\item northmost station
\item peaks prior to 1980
\item pretty consistently lower after 1980
\item zooming in to (0,10), erratic, harder to pick out a yearly "cycle"
\end{itemize}
\subsection{Parsimonious Model}
\begin{itemize}
\item day of year, a slight "bump", not great in 1985-1995ish region
\item more wiggle in data.dec
\item pheo erratic, especially bad for 1980-1995ish
\item tn attempts to get downward peakedness, not really successful, often opposite direction of peak of data
\item no real variation in do.per
\end{itemize}
\subsection{Full Model}

\begin{itemize}
\item no variation in sio2
\item same for tp, nh4
\item tss: some attempts at downward spikes, don't really match
\end{itemize}

\section{D10}
\subsection{Data}
\begin{itemize}
\item in the middle of the main line of stations
\item lots of peaks prior to 1990
\item tapers off after 1990, a few big peaks 1994ish, 2014ish
\item zoom in (0,15) to look at 1990 on: "resurgence" after 2000, 1990s are erratic
\end{itemize}
\subsection{Parsimonious Model}
\begin{itemize}
\item day of year pattern doesn't really match up with what is going on
\item noticeable wiggle in data.dec
\item pheo captures the high peak pattern really well, doesn't help observations catch low dips though
\item do.per does decently prior to 1980, otherwise pretty erratic
\end{itemize}
\subsection{Full Model}
\begin{itemize}
\item sal: doesn't match much until after 2000, sort of matches peakedness
\end{itemize}


\section{D12}
\subsection{Data}
\begin{itemize}
\item middle station
\item lots of big peaks up through 1990, 1993, otherwise big drop off
\item zoom in (0,20), bigger peaks in 2000s than late 1990s
\end{itemize}
\subsection{Parsimonious Model}
\begin{itemize}
\item day of year pattern plateaus when there are peaks
\item noticeable wiggliness early on, flattens out after 2000
\item pheo picks up the peaks early on but has less variation/does less well after 2000
\item do.per, not a lot of variation
\end{itemize}
\subsection{Full Model}
\begin{itemize}
\item sal: no real variation
\end{itemize}
\section{D19}
\subsection{Data}
\begin{itemize}
\item towards the east
\item gap of data mid 90s, 2005ish
\item many peaks up until 1990, 1993ish, otherwise much lower values
\item zoom in (0,15) after 2010, better than 2005-2010 period, peak around 2008
\end{itemize}
\subsection{Parsimonious Model}
\begin{itemize}
\item erratic day of year pattern
\item "wiggles" more spread out before the break than after
\item pheo follows same pattern of magnitude as chl both before and after the break
\item variation in tn but it doesn't seem to match anything in chl
\item do.per matches peaks decently before the break and the lack of peaks after the break
\end{itemize}
\subsection{Full Model}
\begin{itemize}
\item more variation in sio2 than we have seen previously, still not a lot (on the order of magnitude of do.per)
\item no variation in tp
\item a little variation but doesn't really match anything, smaller in magnitude than sio2/do.per
\item nh4 does a good job of capturing the "cyclic" structure, better at matching magnitude after the break than before
\end{itemize}

\section{D22}
\subsection{Data}
\begin{itemize}
\item middle station
\item not very big before 1995 but then even lower after then, with a spike in 2013ish
\item zoom in (0, 20): nothing much to note here, after 2010, chl seems increase a bit
\end{itemize}
\subsection{Parsimonious Model}
\begin{itemize}
\item this "M" shape yearly trend that we have seen before, more successful in matching between 2000 and 2010, doesn't quite match the magnitude after 2010 (again showing some increase in chl in this period)
\item data.dec wiggles take longer until about 1990 where the wiggles are a bit more compressed
\item pheo does a good job early on, not so hot 1980s-90s, becomes erratic after 2000
\item do.per is just erratic but does have some variation
\end{itemize}
\subsection{Full Model}
\begin{itemize}
\ite no additional variables available for the full model
\end{itemize}

\section{D26}
\subsection{Data}
\begin{itemize}
\item station towards the east (but still in the middle portion latitude wise)
\item not a lot of chl in general, but much less after 1995
\item zoom (0, 50) confirms above
\item zoom (0, 20) can still sort of see the yearly structure after 2000, relative peaks 2003 and 2013/4ish
\end{itemize}
\subsection{Parsimonious Model}
\begin{itemize}
\item day of year: clearly defined "M" shape, breaks matching 1985-95ish, seems to be off center in matching the peaks after 2000
\item wide wiggles in data.dec until 2000 and then pretty flat
\item not one of the better pheo series I've seen, no real variation after 1995, erratic prior to that
\item very low magnitude erratic variation in tn
\item do.per on the order of magnitude of pheo but maintains magnitude pretty much throughout
\end{itemize}
\subsection{Full Model}
\begin{itemize}
\item no real variation in sio2
\item no variation in tp, completely flat
\item no real variation in tss
\item definite "cyclical" structure, matches peak location, not magnitude
\end{itemize}

\section{D28A}
\subsection{Data}
\begin{itemize}
\item station towards the east (starting down latitude wise)
\item pretty high peaks until mid 90s, tapers off with a few peaks in 1999 and 2001 (ish)
\item zoom in (0, 20): same spread 1995 to the end, don't really see an uptick after 2010
\end{itemize}
\subsection{Parsimonious Model}
\begin{itemize}
\item day of year more of a peak than "M" shape, gets off track matching wise 1985-2000ish
\item very wiggly data.dec, picks up magnitude pretty well
\item pheo has variation that matches decently until 1995ish, then decreases in magnitude
\item no real variation in tn
\item do.per like pheo but with less matching early on
\end{itemize}
\subsection{Full Model}
\begin{itemize}
\item no real variation in sio2
\item flat for tp
\item no real variation in tss
\item structure in nh4, matches "cyclic" structure pretty well, not so well on magnitude
\end{itemize}

\section{D4}
\subsection{Data}
\begin{itemize}
\item center station
\item big peaks until the end of 1980s, peak 1993ish, 2013ish
\item zoom in (0,20): really low at the beginning of the 90s before the peak, then pretty consistent after the 1993ish peak
\end{itemize}
\subsection{Parsimonious Model}
\begin{itemize}
\item biggest "M" structure (probably just scale), don't match magnitude until after 1995, tracks minimum in, underestimates it early on
\item pretty wiggly data.dec, matches magnitude more after 1990
\item pheo: lots of variation, maintains magnitude throughout, underestimates a lot towards the end of the series, matches well before 1980
\item no real variation in tn
\item erratic variation in do.per, smaller order of magnitude than pheo
\end{itemize}

\subsection{Full Model}
\begin{itemize}
\item decent amount of variation in sio2, order of magnitude as do.per from parsimonious model, matches better after 1995
\item no real variation in tp, a few peaks that don't really match
\item same for tss
\item nh4: small order of magnitude variation, matches better after 1995, severely underestimates before 1990
\item salinity: variation on a decent order of magnitude, shape matches more after 1995, magnitude underestimates overall
\end{itemize}

\section{D41}
\subsection{Data}
\begin{itemize}
\item far west station
\item small amount of chl throughout, no real change in magnitude throughout

\end{itemize}
\subsection{Parsimonious Model}
\begin{itemize}
\item day of year: plateau cyclical structure, dips track lows
\item no wiggle in data.dec
\item pheo: not as useful as before tracking wise, magnitude of variability drops off after 2000
\item no real variability in tn
\item do.per: same order of magnitude as pheo, doesn't seem to track much
\end{itemize}
\subsection{Full Model}
\begin{itemize}
\item sio2: flat line
\item same for tp
\item neglible variation for tss
\item nh4: same order of magnitude as do.per, doesn't seem to track anything
\item sal: same behavior as nh4
\end{itemize}

\section{D6}
\subsection{Data}
\begin{itemize}
\item towards the west
\item not very much chl, but drops off even more around 1990
\item zoom (0,10): doesn't really reveal anything else
\end{itemize}
\subsection{Parsimonious Model}
\begin{itemize}
\item doy: magnitude doesn't match until around 2000, breaks down matching center of peaks end of 1980s-beginning of 1990s
\item very wiggly data.dec
\item pheo: more variation until about 1990 then tapers off with a peak in 1993, tracks decently before 1980
\item no real variation in tn
\item do.per: same order of magnitude as pheo, doesn't really track anything
\end{itemize}
\subsection{Full Model}
\begin{itemize}
\item sio2: a little smaller magnitude than do.per in parsimonious model, no obvious tracking
\item no real variation in tp
\item same for tss
\item nh4: like sio2
\item sal: a little more structure, tracks the peaks, does better after 2000
\end{itemize}


\section{D7}
\subsection{Data}
\begin{itemize}
\item midwest station, slightly higher latitude wise
\item decent magnitude until the end of the 1990s, peak in 2000, relative peaks after 2010
\item zoom in (0, 20): does seem to be an increase after 2010
\end{itemize}
\subsection{Parsimonious Model}
\begin{itemize}
\item doy: squiggle more than the "M" shape, underestimates everything
\item really wiggly data.dec in 1990-2005, fairly straight before 1990
\item pheo: lots of variation early on, smaller magnitude after 1990, underestimates everything
\item no real variation in tn, but some peaks especially in the 2000s
\item  do.per: more consistent magnitude, smaller than the start of pheo, doesn't seem to track anything
\end{itemize}
\subsection{Full Model}
\begin{itemize}
\item no real variation in sio2
\item same for tp
\item tss: same order of magnitude of do.per before, doesn't really track anything
\item no real variation in nh4
\item no real variation in sal except for end of 80s-beginning of 90s and after 2000 (plateau structure, seems to track peaks)
\end{itemize}
**NOTE** Figure out this underestimation thing. Artifact or lack of fit?
It looks like the pre-1990 is a "step" above the 1990s, and then the 2000s are between the 90s level and the 80s level, so I can see why the "intercept" wouldn't really center this.
\section{D8}
\subsection{Data}
\begin{itemize}
\item center station
\item plenty of reasonable sized peaks before the end of the 80s, then a drop off with relative peaks around 1993, 2000, and 2013
\item zoom in (0,20): seems to be a bit of resurgence after 2000
\end{itemize}
\subsection{Parsimonious Model}
\begin{itemize}
\item day of year tracks peaks, magnitude matches better after 2000
\item wide wiggles in data.dec
\item pheo:  matches really well until about 1985, magnitude dampens around 1990
\item tn: flat
\item very little variability in do.per
\end{itemize}
\subsection{Full Model}
\begin{itemize}
\item not a large magnitude of variability for sio2 but does seem to track peaks
\item flat tp
\item flat tss
\item small magnitude of variability for nh4 but does seem to track peaks
\item sal: small variability, tracks more obviously after 2000
\end{itemize}

\section{MD10}
\subsection{Data}
\begin{itemize}
\item far east station, still on the center portion latitude wise
\item peaks all throughout with a few lulls in the late 90s, smaller magnitude from the late 2000s on
\end{itemize}
\subsection{Parsimonious Model}
\begin{itemize}
\item very small magnitude for day of year pattern
\item very wiggly data.dec, more wiggling from the 2000s on
\item pheo tracks well throughout and keeps the same order of magnitude throughout
\item very small magnitude of variability for tn, seems to track lows
\item do.per has a decent magnitude of variability but doesn't seem to track until the 2000s
\end{itemize}
\subsection{Full Model}
\begin{itemize}
\item no real variability in sio2
\item same for tp
\item very small magnitude of variability for tss
\item basically flat for nh4
\end{itemize}

\section{P8}
\subsection{Data}
\begin{itemize}
\item far east station and slightly lower in latitude
\item low values of chl except for prior to 1980
\item resurgence at the beginning of the 2000s, but then it tapers off again
\item zoom in (0, 25): nothing more to add, confirms what is going on in the 2000s
\end{itemize}
\subsection{Parsimonious Model}
\begin{itemize}
\item smooth peak pattern for day of year
\item very little curvature in data.dec
\item relatively small magnitude of variation for pheo, doesn't seem to track much
\item very little variation in tn
\item small magnitude of variation for data.dec (sort of like pheo), doesn't seem to track much
\end{itemize}
\subsection{Full Model}
\begin{itemize}
\item very little variation in sio2
\item same for tp
\item same for tss
\item same for nh4
\end{itemize}
**NOTE** Looks like there is a "wave" in the chl data which I think is going to mess up the "centering" that the intercept is adding to the components.

\section{Some Other Things}

\begin{itemize}
\item Sometimes there is a straight line jump in the late 90s. I tracked this pack to do.per being missing in this region.
\item The fitted values for the parsimonious model are often better than those of the full model. This is not surprising since we mainly added variables into the full model because we are interested in them in context. The variables that looked to have a strong correlation to chl were included in the parsimonious model.
\item I considered trying to smooth the components of each model, but I decided that would add too much subjective ``fiddling".
\item Next steps: go back and look on the monthly scale, see if peak location in a year changes over time. Along these lines, I want to go back and fit quick models with just day of year and dec.time along with their interaction to allow the day of year trend to change over time. 
\item Moving forward to the spatial model: parsimonious model but probably drop do.per. We see here that it often is similar to pheo but not as good.
\item But we are still interested in all of these other variables even though they don't contribute much to the prediction of chlorophyll. Might need to take a step back and do simpler aggregations of the variables and compare to get at those questions. There could also be lag, but we didn't see any obvious evidence for this from the positions of the component predictions.
\end{itemize}

\section{Interaction Model}

\begin{itemize}
\item Stations where interaction term changes magnitude over time: C10, D10, D12, D22, D4, D7, D8
\item Stations where interaction term seems fairly constant over time: C3, D41, D6, P8
\item Weird behavior: D19, D26, D28A, MD10
\end{itemize}
\end{document}