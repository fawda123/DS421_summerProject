
\documentclass[12pt]{amsart}
\usepackage{geometry} % see geometry.pdf on how to lay out the page. There's lots.
\geometry{a4paper} % or letter or a5paper or ... etc
% \geometry{landscape} % rotated page geometry

% See the ``Article customise'' template for come common customisations

\title{Observations from Per Station Models}
\author{Sara Stoudt}
%\date{} % delete this line to display the current date

%%% BEGIN DOCUMENT
\begin{document}

\maketitle
%\tableofcontents

\section{C10}
\subsection{Data}
\begin{itemize}
\item southmost station
\item peaks before 1975, 1976, 1991,1992, 2008, 2009
\item can clearly see year structure except for 1982, 1983,1995, 1998,2005, 2006, 2012
\end{itemize}

\subsection{Parsimonious Model}

\begin{itemize}
\item day of year captures clear year structure
\item pheo also helpful with clear year structure, magnitude reflects magnitude of chl
\item tn not super helpful
\item do_per like pheo although a little less consistent with the magnitude matching
\end{itemize}

\subsection{Full Model}

\begin{itemize}
\item sio2 no real variation 
\item same for tp and tss
\item nh4: a bit of peakedness, seems to lag a bit from the peak of chl, relationship lost after 1995

\end{itemize}

\section{C3}
\subsection{Data}
\begin{itemize}
\item northmost station
\item peaks prior to 1980
\item pretty consistently lower after 1980
\item zooming in to (0,10), erratic, harder to pick out a yearly "cycle"
\end{itemize}
\subsection{Parsimonious Model}
\begin{itemize}
\item day of year, a slight "bump", not great in 1985-1995ish region
\item more wiggle in date_dec
\item pheo erratic, especially bad for 1980-1995ish
\item tn attempts to get downward peakedness, not really successful, often opposite direction of peak of data
\item no real variation in do_per
\end{itemize}
\subsection{Full Model}

\begin{itemize}
\item no variation in sio2
\item same for tp, nh4
\item tss: some attempts at downward spikes, don't really match
\end{itemize}

\section{D10}
\subsection{Data}
\begin{itemize}
\item in the middle of the main line of stations
\item lots of peaks prior to 1990
\item tapers off after 1990, a few big peaks 1994ish, 2014ish
\item zoom in (0,15) to look at 1990 on: "resurgence" after 2000, 1990s are erratic
\end{itemize}
\subsection{Parsimonious Model}
\begin{itemize}
\item day of year pattern doesn't really match up with what is going on
\item noticeable wiggle in date_dec
\item pheo captures the high peak pattern really well, doesn't help observations catch low dips though
\item do_per does decently prior to 1980, otherwise pretty erratic
\end{itemize}
\subsection{Full Model}
\begin{itemize}
\item sal: doesn't match much until after 2000, sort of matches peakedness
\end{itemize}


\section{D12}
\subsection{Data}
\begin{itemize}
\item middle station
\item lots of big peaks up through 1990, 1993, otherwise big drop off
\item zoom in (0,20), bigger peaks in 2000s than late 1990s
\end{itemize}
\subsection{Parsimonious Model}
\begin{itemize}
\item day of year pattern plateaus when there are peaks
\item noticeable wiggliness early on, flattens out after 2000
\item pheo picks up the peaks early on but has less variation/does less well after 2000
\item do_per, not a lot of variation
\end{itemize}
\subsection{Full Model}
\begin{itemize}
\item sal: no real variation
\end{itemize}
\section{D19}
\subsection{Data}
\begin{itemize}
\item towards the east
\item gap of data mid 90s, 2005ish
\item many peaks up until 1990, 1993ish, otherwise much lower values
\item zoom in (0,15) after 2010, better than 2005-2010 period, peak around 2008
\end{itemize}
\subsection{Parsimonious Model}
\begin{itemize}
\item erratic day of year pattern
\item "wiggles" more spread out before the break than after
\item pheo follows same pattern of magnitude as chl both before and after the break
\item variation in tn but it doesn't seem to match anything in chl
\item do_per matches peaks decently before the break and the lack of peaks after the break
\end{itemize}
\subsection{Full Model}
\begin{itemize}
\item more variation in sio2 than we have seen previously, still not a lot (on the order of magnitude of do_per)
\item no variation in tp
\item a little variation but doesn't really match anything, smaller in magnitude than sio2/do_per
\item nh4 does a good job of capturing the "cyclic" structure, better at matching magnitude after the break than before
\end{itemize}
\section{D22}
\subsection{Data}
\begin{itemize}
\item 
\end{itemize}
\subsection{Parsimonious Model}

\subsection{Full Model}

\section{D26}
\subsection{Data}
\begin{itemize}
\item
\end{itemize}
\subsection{Parsimonious Model}

\subsection{Full Model}

\section{D28A}
\subsection{Data}
\begin{itemize}
\item
\end{itemize}
\subsection{Parsimonious Model}

\subsection{Full Model}

\section{D4}
\subsection{Data}
\begin{itemize}
\item
\end{itemize}
\subsection{Parsimonious Model}

\subsection{Full Model}

\section{D41}
\subsection{Data}
\begin{itemize}
\item
\end{itemize}
\subsection{Parsimonious Model}

\subsection{Full Model}

\section{D6}
\subsection{Data}
\begin{itemize}
\item
\end{itemize}
\subsection{Parsimonious Model}

\subsection{Full Model}

\section{D7}
\subsection{Data}
\begin{itemize}
\item
\end{itemize}
\subsection{Parsimonious Model}

\subsection{Full Model}

\section{D8}
\subsection{Data}
\begin{itemize}
\item
\end{itemize}
\subsection{Parsimonious Model}

\subsection{Full Model}

\section{MD10}
\subsection{Data}
\begin{itemize}
\item
\end{itemize}
\subsection{Parsimonious Model}

\subsection{Full Model}

\section{P8}
\subsection{Data}
\begin{itemize}
\item
\end{itemize}
\subsection{Parsimonious Model}

\subsection{Full Model}

\end{document}